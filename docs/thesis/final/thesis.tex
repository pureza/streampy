\documentclass[a4,11pt]{report}


\usepackage{alltt}
\usepackage{graphicx}
\usepackage{color}
\usepackage[small,bf]{caption}
\usepackage{listings}
\usepackage{cite}
\usepackage{url}
\usepackage[hmargin=3.5cm]{geometry}

\lstdefinelanguage{EzQL}
{
  morekeywords=[1]{from, select, where, group, by, in, hour,
    entity, match, with, if, then, else, end, and, stream, of,
    createFrom, hasMany, belongsTo, let, in, min, into, member,
    when, enum},
  morekeywords=[2]{avg, last, max, values, minBy, howLong, any,
    changes, count},
  sensitive=false,
  morecomment=[l]{\#},
  morestring=[b]",
  tabsize=2,
}

\lstdefinelanguage{CCL}
{
  morekeywords=[1]{dummy, from, select, where, group, by, create, input,
    stream, schema, string, integer, as, window, insert, removed, rows,
  keep, seconds, into, float, last, per, outer, join, boolean,
  if, then, else, end, dummy, ON, left, right, interval, timestamp,
  full, local, and, hours, minutes, dummy}
  morekeywords=[2]{dummy, max, coalesce, now, avg, dummy},
  sensitive=false,
  morecomment=[l]{--},
  morestring=[b]",
}

\lstdefinelanguage{Oracle}
{
  morekeywords={dummy, on, if, then, and, dummy}
  sensitive=false,
  morecomment=[l]{--},
  morestring=[b]",
}


\lstdefinelanguage{MonitorScript}
{
  morekeywords=[1]{dummy, monitor, dictionary, string, int, action, on, all,
    route, if, then, dummy}
  morekeywords=[2]{},
  sensitive=false,
  morecomment=[l]{//},
  morestring=[b]",
}



\definecolor{light-gray}{gray}{0.97}
\definecolor{dark-gray}{gray}{0.50}

\begin{document}
\title{EzQL: A new language for Event Stream Processing}
\author{Lu\'{\i}s Pureza}

\maketitle

\tableofcontents

\addtolength{\parskip}{\baselineskip}
\chapter{Introduction}
\label{chap:introduction}

\lstset{
  language=EzQL,
  columns=fullflexible,
  basicstyle=\tt,
  keywordstyle=[1]\bf,
  keywordstyle=[2]\it,
  % frame=leftline,
  % framerule=1pt,
  % rulesep=20pt,
  % backgroundcolor=\color{light-gray},
  % rulecolor=\color{dark-gray}
}

\section{Streams and windows}


As we will see, EzQL is very different from the query languages
presented in the state of the art. However, at its core, it still has
streams of events and operations to query them.

Suppose you are given a stream called \verb=stocks= that receives
events from the financial market. These events contain three fields --
\verb=timestamp=, \verb=symbol= -- which is the company id --, and
\verb=price=. In EzQL you would begin by declaring this stream with:

\begin{lstlisting}
  stocks = stream of { timestamp:int, symbol:string, price:float }
\end{lstlisting}

Now that the stream is ready to use, you may query its data. For
example, to multiply the price of ACME events by two, you could write:

\begin{lstlisting}
  acmeStocks_x2 =
    from ev in stocks
    where ev.symbol == "ACME"
    select { price_x2 = ev.price * 2 }
\end{lstlisting}

The result of this operation is a new stream -- \verb=acmeStocks_x2=
--, that can be subjected to further analysis. The next listing shows
off a few more features that EzQL shares with other languages, namely
temporal windows and aggregations:

\begin{lstlisting}
  avg_5min =
    let cheapStocks =
      from ev in stocks[5 min]
      where ev.price < 20
    in cheapStocks.avg (:price)
\end{lstlisting}

This example computes the average price of all events received during
the last 5 minutes where the price is under \$20. The query has been
partitioned into two smaller sub-queries: the first --
\verb=cheapStocks= --, which is declared (using \verb=let=) as a local
stream containing the desired events, and the second which is the
application of the \verb=avg()= method over \verb=cheapStocks=.

In traditional SQL or derived languages, \verb=avg_5min= would be a
stream, whose events would represent changes to the average. However,
as we have seen in chapter ??, events are ephemeral and if we need the
current average to use it in other queries, we have to keep it inside
a window with a single entry created specifically for this
purpose. Yet, conceptually, the average is just a value that should be
available directly. Hiding this value inside a window represents an
unnecessary level of indirection that impacts the amount of code that
needs to be written and hides its meaning.  We believe this is an
instance where the language semantics do not map well with the way
programmers think. What we need here is a variable to store the
average, not streams or windows.

Thus, in EzQL, \verb=avg_5min= is a simple number whose value is
automatically updated by the engine as the average changes. This
number can be accessed in subsequent queries by referencing its name
-- \verb=avg_5min=. For example, to check if it is smaller than 10 we
could write:

\begin{lstlisting}
  isSmall = avg_5min < 10
\end{lstlisting}

While it may seem uninteresting and purely fictional, this example
illustrates one important point: \verb=isSmall= is a continuously
running query. Every time the average is updated, so is
\verb=isSmall=. The engine works like a spread-sheet application where
a modification to one cell is automatically propagated to other cells
that depend on the first. Notice how \verb=avg_5min= has been
completely abstracted from its definition that depends on the
\verb=stocks= stream. The developer may simply use \verb=avg_5min=
without needing to worry himself with the low-level details of how and
when this value will be updated. As we will see, this is a crucial
ability that allows the language to grow, by enabling developers to
write more complex queries while retaining conciseness and
expressiveness.

Values such as \verb=isSmall= and \verb=avg_5min=, that are always
defined and are managed by the engine, are called \emph{continuous
  values}, in opposition to streams which are discrete. Continuous
values play an important role in EzQL and their usage will become
ubiquitous in the rest of this chapter.

\section{Dictionaries}

The next example shows how to obtain the current price per company:

\begin{lstlisting}
  pricePerCompany =
    from stocks
    group by :symbol into g
    select g.last().price
\end{lstlisting}

Group by works by partitioning the events from the input stream into N
smaller streams, one per group. In the example above, each one of
these streams is given the name \verb=g= and then, for each one in
parallel, the price from its last event is retrieved -- using the
\verb=last()= method --, to be used as the resulting value of the
query.

In this listing, \verb=pricePerCompany= is a dictionary that maps keys
(the company symbols) to values (the current price of their
stocks). Moreover, being the result of a continuous query,
\verb=pricePerCompany= is also a continuous entity: when a new event
arrives, the price of its company is automatically updated and every
time a previously unseen symbol is processed, a new entry is created
for it. Moreover, these changes are automatically propagated to other
queries that depend on them, as can be seen in the next example:

\begin{lstlisting}
  cheapCompanies =
    from price in pricePerCompany
    where price < 20
    select price
\end{lstlisting}

\verb=cheapCompanies= is a filtered dictionary: it contains only those
companies whose current price is below \$20. Every time a company goes
above this threshold, it is removed from the dictionary; and every
time the stocks of some company descend below \$20, a new entry is
created for it.

Sometimes one needs to return multiple values in the same query. For
example, one might need to return the maximum and average prices over
the last 5 minutes for all companies. In EzQL, this can be done as
follows:

\begin{lstlisting}
  avgMax_5min =
    from stocks
    group by :symbol into g
    select let g5 = g[5 min] in
           { maximum = g5.max(:price), average = g5.avg(:price) }
\end{lstlisting}

The segment in braces creates a record with two attributes:
\verb=maximum= and \verb=average=. One can access individual
attributes in a record using the dot (\verb=.=) operator. For example,
it is possible to access ACME's maximum price with:

\begin{lstlisting}
  acme_max = avgMax_5min["ACME"].maximum
\end{lstlisting}

This example also shows how to index individual entries inside a
dictionary using square brackets. Needless to say, whenever a new
event is received for ACME, or an old event leaves the 5 minutes
window, \verb=acme_max= will automatically be updated to reflect this
change.

To conclude this section, we show how to perform aggregations over
dictionaries. For example, to find the company with the cheapest
stocks overall, you could write:

\begin{lstlisting}
  cheapest =
    let companies =
      from stocks
      group by :symbol into g
      select g.last()
    in companies.values().minBy(:price).symbol
\end{lstlisting}

\verb=companies= is a dictionary that contains the last event received
for each company. Remember that a dictionary maps keys to values. The
method \verb=values()= allows us to obtain a list containing just the
values which, in this case, are the events. To this list, we apply the
\verb=minBy()= aggregator which returns the event with the lowest
price. Finally, we access the \verb=symbol= field of this event to
obtain the corresponding company id.

\section{Continuous values}

We have already discussed continuous values in section ??. This
section will expand that knowledge by showing what is possible to do
with them.

Being, by definition, values that change, it is only natural to
exploit the temporal information contained in them. As such, it is
possible to open windows and apply aggregators over regular integers,
booleans or other values. The following listing illustrates this:

\begin{lstlisting}
  maxPerCompany_5min =
    from stocks
    group by :symbol into g
    select g.last().price[1 hour].max()
\end{lstlisting}

The result is a dictionary that keeps, for each company, the maximum
price its stocks attained over the course of the last hour.

Note that the 1 hour window is created over \verb=g.last().price= --
that is, the current price. You could have solved this by opening a
regular event window over the stream \verb=g=, as in section
\ref{sec:stream-operations}:

\begin{lstlisting}
  maxPerCompany_1h =
    from stocks
    group by :symbol into g
    select g[1 hour].max(:price)
\end{lstlisting}

Surprisingly though, the results would differ. To see why, note that
this is just another instance of the pulse vs state-changes problem
discussed in section \ref{sec:acme-problem}: the price could have been
at its maximum 1 hour ago, but the notification for this value might
have happened a little earlier and thus, a regular sliding window
applied to a stream would not consider it when calculating the
maximum. Continuous values, on the other hand, behave differently. As
mentioned in section \ref{sec:stream-operations}, an object's
attribute retains its value until a new event changes it. Thus,
querying the attribute for its maximum value over the last hour would
consider all the values the attribute possessed during that time ---
even if the corresponding event came earlier --- and would give the
correct result.

This is surprising because continuous values were introduced to
simplify queries. Nonetheless, they also introduce state and this
seemingly innocuous change has profound implications. In essence,
events are discrete while continuous values are just that:
continuous. Through them, we can now support the state-changes
semantics discussed in chapter
\ref{chap:simple-questions-complex-answers} while retaining support
for pulse semantics by using regular stream operations.

Continuous values are a natural medium to express a few recurring
patterns. For example, a common problem in ESP asks if a given
condition was true any time or every time during some period of
time. In EzQL, this can be done with the \verb=any?()= or
\verb=all()?=\footnote{EzQL allows functions that end in '?'; a
  convention, found in other languages such as Scheme and Ruby, that
  indicates that the function is a predicate.}  aggregators:

\begin{lstlisting}
  isSmall =
    let acmeMax = maxPerCompany_1h["ACME"] in
    acmeMax < 20

  alwaysSmall = isSmall[1 hour].all?()
\end{lstlisting}

In the previous example, \verb=isSmall= is a boolean value that is
true if the maximum price over the last hour for ACME is smaller than
\$20. Since this maximum will vary, \verb=isSmall= is expected to
alternate between \verb=true= and \verb=false=. To check if the
maximum was always small, we just need to open a 1 hour window over
\verb=isSmall= and use \verb=all?()= over that window.

A related problem asks for the duration of a given condition. In EzQL,
we can solve it using the \verb=howLong()= aggregator\footnote{50 min
  means 50 minutes}

\begin{lstlisting}
  mostlySmall = isSmall[1 hour].howLong() > 50 min
\end{lstlisting}

\section{Modeling entities}

\label{sec:entities}

\subsection{Motivation}

In chapter \ref{chap:simple-questions-complex-answers}, we argued that
some queries would become simpler if the developer could model the
state of each entity and the relationships between them using
objects. In this section we will explain how to do this in EzQL using
a simple fictional scenario based on a factory. This factory has many
rooms and each room is crossed by a pipeline that carries products
between rooms. Furthermore, each room has one temperature sensor --
because the conditions under which the products are manufactured must
be rigorously controled --, and one RFID sensor that signals the
entrace of a new product in the room. We can model this scenario using
two streams:

\begin{itemize}
\item \verb!tempReadings = stream of { roomId:string, temperature:float }! which receives
  temperature readings per room;
\item and \verb!entries = stream of { roomId:string, productId:int }! whose events are
  created when a product has entered a room.
\end{itemize}

For this particular use-case, it might be appropriate to set up a
dictionary with all the room's data and another with the products':

\begin{lstlisting}
  allRooms =
    from tempReadings
    group by :roomId into g
    select let ev = g.last() in
           { roomId = ev.roomId, temperature = ev.temperature }

  allProducts =
    from entries
    group by :productId into g
    select let ev = g.last() in
           { productId = ev.productId, roomId = ev.roomId }
\end{lstlisting}

With these dictionaries, one can, for example, find out which products
are in rooms where the temperature is greater than 30 degrees:

\begin{lstlisting}
  hotProducts =
    from p in allProducts
    where allRooms[p.roomId].temperature > 30
\end{lstlisting}

Or how many products are in each room:

\begin{lstlisting}
  productsPerRoom =
    from r in allRooms
    select let prodsInR =
             from p in allProducts
             where p.roomId == r.roomId
           in prodsInR.count()
\end{lstlisting}

While this approach works, it also feels rather unnatural. Basically,
\verb=allRooms= and \verb=allProducts= are nothing more than
containers for the current state of all rooms and products known by
the system. This current state is very easy to manage and if you look
closer at their queries, you'll see that they have a lot in common:
the only things that change are the attribute names. Furthermore, the
queries for \verb=hotProducts= and \verb=productsPerRoom= follow
directly from the fact that there is an association between rooms and
products: at any given moment, a product is in one room while each
room may contain any number of products. As a consequence, these
queries also exhibit a few common patterns that could be abstracted.

\subsection{Modeling simple entities}

To improve this situation, EzQL includes an entity framework inspired
by Object-Relational Mapping tools for databases, that allows the
developer to model the entities that make up a scenario, as well as
the associations between them. This results in simpler queries
compared to the ones above. For example, defining an entity to model
rooms can be done as follows:

\begin{lstlisting}
  entity Room =
    createFrom (tempReadings, :roomId)
\end{lstlisting}

This is equivalent to the definition of \verb=allRooms=
above. \verb=createFrom= instructs the system to automatically create
new \verb=Room= entities whenever a previously unseen \verb=roomId=
arrives on \verb=tempReadings=. In addition, \verb=createFrom= adds
the stream fields to the class. This way, every Room instance ends up
with two attributes: \verb=roomId= and \verb=temperature=. These
attributes are automatically filled and updated by the system as new
events arrive, so that developers don't have to worry with such
details.

Having declared \verb=Room=, we can now run a simulation to see how
the system manages these entities automatically. All the \verb=Room=
instances are accessible using \verb=Room.all=. This is a dictionary
that maps \verb=Room= id's (taken from the \verb=roomId= field in
this case) to the \verb=Room=s themselves. In essence, it plays a similar
role to \verb=allRooms= above. If the stream \verb=tempsReadings=
receives the following events:

\begin{tabular}{ |l|l|c| }
  \hline
  \verb=timestamp= & \verb=roomId= & \verb=temperature= \\
  \hline
  11:00:00 am & \verb="A"= & 20 \\
  11:01:00 am & \verb="B"= & 29 \\
  11:02:00 am & \verb="A"= & 23 \\
  11:03:00 am & \verb="B"= & 30 \\
  \hline
\end{tabular}

then \verb=Room.all=, if evaluated after 11:03:00 am, will return:

\begin{tabular}{ |l|l| }
  \hline
  Entity id & Entity \\
  \hline
  \verb="A"= & Room with roomId = \verb="A"=, temperature = 23 \\
  \verb="B"= & Room with roomId = \verb="B"=, temperature = 30 \\
  \hline
\end{tabular}

Naturally, we can query these entities as if they were normal streams:

\begin{lstlisting}
  hotRooms =
    from   room in Room.all
    where  room.temperature > 25
    select room.roomId
\end{lstlisting}

This results in a subset of \verb=Room.all= containing only one room:

\begin{tabular}{ |l|l| }
  \hline
  Entity id & Entity \\
  \hline
  \verb="B"= & Room with room\_id = \verb="B"=, temperature = 30 \\
  \hline
\end{tabular}

\subsection{Modeling associations}

Modeling associations between entities is just as easy. For instance,
to specify the one-to-many relationship between products and rooms we
may redefine the entities as follows:

\begin{lstlisting}
  entity Room =
    createFrom (tempReadings, :roomId)
    hasMany :products

  entity Product =
    createFrom (entries, :productId)
    belongsTo :room
\end{lstlisting}

As the name implies, \verb=hasMany= means that each room may contain
an unbounded number of products. This annotation also adds a
collection of \verb=Product= instances to every \verb=Room=. So, the
list of products in room A may be retrieved with:

\begin{lstlisting}
  Room.all["A"].products
\end{lstlisting}

In fact, \verb=products= is not a list, but a dictionary defined by
taking only those products that belong to room A from
\verb=Product.all=. As such, we can apply other dictionary operators
to this attribute. To show this, the next example finds which products
that are now in room A were also in room B less than 5 minutes ago:

\begin{lstlisting}
  from p in Room.all["A"].products
  where (p.roomId == "B")[5 min].any?()
\end{lstlisting}

The predicate opens a window over the result of \verb!p.roomId == "B"!
and the \verb=.any?()= aggregator checks if the value was true any
time during that interval.

This example also features a previously unseen situation where the
predicate requires historical data in order to actuate. This poses
some difficulties because the set of products that are in room A is
constantly changing and, in order to work correctly, the system must
take that into consideration. That is, as soon as a product arrives
into room A, its 5 minute data must already be available. The
implementation of this class of queries -- that combine filtered
dictionaries with temporal windows --, presents some challenges that
will be discussed in chapter ??.

Proceeding with the previous example, \verb=belongsTo= sets up the
reverse association, i.e., every product is inside one room. Moreover,
this instruction also adds an attribute \verb=room= to every
\verb=Product=. Thus, you could find in which room product 123 is
with:

\begin{lstlisting}
  Products.all[123].room
\end{lstlisting}

This may seem like a bit of hidden magic at first, but it's really
just a simple convention:

\begin{enumerate}
\item The \verb=:products= parameter to \verb=hasMany= instructs the
  engine to create a list of \verb=Product= instances. It knows that
  this list should hold \verb=Product=s, because ``product'' is the
  singular of ``products''. This list will be named \verb=products=;
\item On the other hand, the \verb=:room= parameter to
  \verb=belongsTo= tells the engine to create a field named
  \verb=room= in every product. This field will hold an instance of
  the \verb=Room= class (once again, it finds the class name through
  the parameter);
\item The engine fills the \verb=room= attribute of every product by
  checking its \verb=roomId= field. Remember, this field was created
  automatically by \verb=createFrom=. Given the identifier of the
  room, the engine can easily lookup the corresponding \verb=Room=
  instance using the \verb=Room.all= dictionary;
\item The \verb=products= list in every \verb=Room= is filled using
  the reverse relationship. That is, the products that belong to room
  A are those where the condition \verb!room_id == "A"! is true.
\end{enumerate}

\subsection{Defining additional attributes}

It is possible to add new attributes to an entity, on top of those
that are created automatically by the system. For example, suppose we
want to add a temperature attribute to every product that is taken
from the temperature of the room where the product currently is. We
could do it with:

\begin{lstlisting}
  entity Product =
    createFrom (entries, :productId)
    belongsTo :room

    member this.temperature = this.room.temperature

\end{lstlisting}

There is a subtle detail in this definition: despite being equal to
the temperature of the current room, the product's temperature has a
``life of its own'' when it comes to history. That is, the past values
of the product's temperature reflect the fact that the product might
have been in other rooms. Hence, \verb=this.room.temperature[5 min]=
is, in general, different from the current room's temperature over the
last 5 minutes.


\section{Additional features}

\subsection{When blocks}

Event processing problems are all about executing the right actions
when something happens, be it the arrival of a new event or the
expiration of an old one. In EzQL, we have decided to abstract this
notion of \emph{when} to execute an action as much as possible,
because this results in queries that are more declarative and less
algorithmic. If you're not convinced about this, go back to the
previous section, choose any query, and try to enumerate the external
events that, ultimately, trigger the re-evaluation of the entire
query. Allowing the developer to write his queries in a higher-level
of abstraction, without needing to worry with such details (``What
happens if two events arrive at the same time?'', ``What happens if,
at the same time, an event expires and another arrives at a window?'')
should be seen as one of the most important features of EzQL.

Nonetheless, there are situations where specifying the right time to
execute some action is actually the way to go, because it gives the
developer more freedom and allows him to do things that couldn't be
done using the standard operators. For example -- and continuing with
our rooms and products example --, suppose you'd like to know, for
each room, how many products entered the room, but only if the room's
temperature was hot (greater than 25). In EzQL you could employ when
blocks to solve this problem:

\begin{lstlisting}
  entity Room =
    createFrom (tempReadings, :roomId)
    hasMany :products

    member self.hotEntries = 0
      when ev in entries ->
        # If the entry happened in this room and this room is hot...
        if ev.roomId == self.roomId and self.temperature > 25
          then self.hotEntries + 1
          else self.hotEntries


\end{lstlisting}

The snippet above adds a new field called \verb=hotEntries= to the
\verb=Room= entity. When a room instance is created, this field starts
at 0. However, when a product enters a hot room, the \verb=hotEntries=
field for the corresponding room instance will be incremented by 1,
while all the other rooms will remain unchanged.

You may also use when blocks with more than one stream. For example,
reseting the counter when the temperature goes below 25 is a
straightforward extension:

\begin{lstlisting}
    member self.hotEntries = 0
      when
        | ev in entries ->
            if ev.roomId == self.roomId and self.temperature > 25
              then self.hotEntries + 1
              else self.hotEntries
        | self.temperature.changes() if self.temperature < 25 -> 0
\end{lstlisting}

The \verb=changes()= operator was introduced in section ??. Basically,
it converts the attribute \verb=self.temperature= into a stream with
the different values it takes during execution. When an event arrives
into that stream (meaning: when the temperature changes), if the new
temperature is below 25, then \verb=hotEntries= will be reset to
0. Notice that \verb=ev in...= is optional: if the event itself is not
necessary, there is no need to declare it.

The careful reader will notice that these last two examples are the
perfect testbed for a feature that was proposed in section ??:
semantic windows. EzQL does not provide lexical support for these
constructs. Yet, as we can see, when blocks already come with the
power necessary to solve the same problems. Maybe the syntax could be
improved, but it's not like when blocks are difficult to use. Besides,
modifying the syntax is the easy part.

\subsection{Variants and pattern matching}

Sometimes we would like to classify the status of a set of entities
using descriptive names. For example, a room could be categorized as
hot or cold, crowded or empty. We could use strings for this, but
strings are not type-safe (what happens if the programmer makes a
typo?). Variants and pattern matching are two features commonly found
in functional languages that work together to help solve these
problems.

Conceptually, a variant is a value that may take a number of
forms. For example, a variant to classify the room temperature may be
declared with:

\begin{lstlisting}
  enum RoomState =
    | Cold
    | Hot
\end{lstlisting}

We could now define a \verb=state= attribute inside the definition of
the \verb=Room= entity that classifies the room as cold or hot:

\begin{lstlisting}
  member self.state =
    if self.temperature <= 25
      then Cold ()
      else Hot ()
\end{lstlisting}

Now, to find out how many rooms are hot we could write the following query:

\begin{lstlisting}
  hotCount =
    let hotRooms =
      from room in Room.all
      where room.state == Hot ()
    in hotRooms.count()
\end{lstlisting}

Variants may also carry payloads -- i.e., each state may also include
inner state. For example, suppose that we want to know how many
products entered the room while it was hot. This is the exact same
problem that we solved in the previous section, but we are now going
to present an alternative solution. First, let's add a counter to the
hot state:

\begin{lstlisting}
  enum RoomState =
    | Cold
    | Hot of int  # int is the type of the counter
\end{lstlisting}

Now we need to manage this counter inside the definition of the
\verb=state= attribute:

\begin{lstlisting}
  # Is this room hot or cold?
  member self.isHot = self.temperature > 25

  member self.state = null
    when
      | (self.isHot).changes() -> if self.isHot then Hot 0 else Cold
      | ev in entries if ev.roomId == self.roomId ->
          match self.state with
          | Cold -> Cold ()
          | Hot c -> Hot (c + 1)

\end{lstlisting}

%% TODO: Talk about NULL before

The first part of this definition is straightforward: the initial
value is \verb=null= -- because we don't know if the room is hot or
cold in the beginning --, and we alternate between the \verb=Hot= and
\verb=Cold= states when the \verb=isHot= condition changes. Notice
that we pass \verb=0= to the \verb=Hot= state, to initialize the
counter.

In the second part we use pattern matching for the first time. The
\verb=match ... with= operator provides a way to deconstruct a
variant. Basically, if the current value is \verb=Cold=, the next
value will continue to be \verb=Cold=. However, if the current state
is \verb=Hot c= for any counter \verb=c=, then we would like the next
state to be \verb=Hot= with the counter incremented by one.

Now we may use these definitions to find out how many rooms are
\verb=Hot= and have more than 10 entries at the same time:

\begin{lstlisting}
  hotPopulatedCount =
    let hotRooms =
      from room in Room.all
      where match room.state with
              | Hot c if c > 10 -> true
              | _ -> false
    in hotRooms.count()
\end{lstlisting}

Inside \verb=match ... with=, the underscore means ``match anything''.

Arguably, this might not be the best example to illustrate the full
potential of variants and pattern matching. After all, we could have
done the same using the concepts from the last section -- and the
result would probably have been easier. However, the purpose of this
section is to introduce the reader to these new concepts. The real
usefulness of variants and pattern matching will be demonstrated in
the following chapters.

\subsection{Functions}

EzQL allows the developer to create user-defined functions and use
them in queries. The following example implements the factorial
function and uses it to calculate the factorial of the number of
products in room A:

\begin{lstlisting}
  def fact n =
    if n == 0
      then 1
      else n * fact (n - 1)

  factCount = fact (Room.all["A"].products.count())
\end{lstlisting}

When the number of products inside the room changes, the system
automatically reevaluates the entire expression and updates
\verb=factCount=.

In EzQL, functions are not status values. Quite the contrary,
functions are allowed to change and their modifications are propagated
downward to the places where they are used. The best way to illustrate
this is through an example:

\begin{lstlisting}
  def twoHourMax id =
    let prodCount = Room.all[id].products.count() in
    prodCount[2 hour].max()

  maxProductsPerRoom_2h =
    from room in Room.all
    select twoHourMax (room.roomId)
\end{lstlisting}

This snippet calculates the maximum number of products that were in
each room over the past two hours. To avoid duplication, the code
responsible for this calculation has been abstracted inside a function
-- \verb=twoHourMax= --, that receives the id of the room as a
parameter.

The function itself depends on the result of the \verb=count()=. When
\verb=prodCount= changes, the entire body must be reevaluated, because
a new maximum may have been found and we want
\verb=maxProductsPerRoom_2h= to be immediately updated. Furthermore,
the function includes a kind of internal state -- the 2 hours window
--, that may itself trigger the computation of a new value.

The next example combines functions with when blocks to illustrate the
full potential of treating functions as active entities:

\begin{lstlisting}
def sumValues (n:int) =
  let acc = 0
    when n.changes() -> acc + n
  in
    acc;;
\end{lstlisting}

The function \verb=sumValues()= receives an integer\footnote{Type
  declarations are optional in EzQL. In this example we declared n as
  an integer to serve as documentation. If we had ommited this
  declaration, the system would have inferred it anyway.}. Inside the
body, we declare a variable -- \verb=acc=, that will act as an
accumulator. \verb=acc= starts at 0 and then, when \verb=n= changes,
it sums the new \verb=n= into the old \verb=acc=. In practice,
\verb=sumValues()= is equivalent to the built-in \verb=sum()= method
for integers. That is, \verb=Room.all.count.sum()= and
\verb=sumValues (Room.all.count)= would return the same result.

This example showed how functions can be used to implement
user-defined aggregators (UDA). We believe this is an important
feature, because many programs require aggregations that go beyond the
usual \verb=min=, \verb=max=, \verb=avg=, \verb=sum= and
\verb=count=. Chapter ?? will explore this idea with a few more
examples.

\subsection{Other features}


Finally, EzQL supports higher-order functions, i.e., functions that
may passed to or returned from other fuctions:

\begin{lstlisting}
  def deriv (f, dx) =
    fun x -> (f(x + dx) - f(x)) / dx
\end{lstlisting}

\verb=deriv= is a function that receives another function \verb=f= and
a step size \verb=dx= and returns another function
(\verb=fun ... -> ...=) that receives an \verb=x= and calculates the
derivative of \verb=f= at \verb=x=.

Let's use this to calculate the derivative of \(x^2\) when \verb=x=
is the number of products in room A:

\begin{lstlisting}
  dProdCount =
    let derivX2 = deriv (fun x -> x * x, 0.01)
    let prodCount = Room.all["A"].products.count()
    derivX2 (prodCount)
\end{lstlisting}

Higher-order functions are used extensively in EzQL programs, even if,
most of the times, the user is not aware of their presence. For
example, the SQL-like syntax we have been using in some queries is
actually just syntactic sugar that is rewritten by the parser before
evaluation begins. This will be explained in chapter ??.


\subsection{EzQL from a design point of view}

\end{document}

