\chapter{Solutions to a few problems}

This appendix contains solutions to some of the problems discussed in
chapter \ref{chap:simple-questions-complex-answers}. We can't claim
these are the best solutions nor the smallest, but they shouldn't be
too far away. We added a few comments to help the understanding of the
code.

\section{The ACME problem}
\label{sec:acme-problem-solution}

\lstset{
  language=CCL,
  columns=fullflexible,
  basicstyle=\tt,
  keywordstyle=[1]\bf,
  keywordstyle=[2]\it,
}


\begin{lstlisting}
-- Algorithm:
--
-- Keep two windows: one for the last 20 minutes of data, and another
-- with the newest event that is older than 20 minutes (i.e., the one
-- that expired last from the first window). Then, join both windows
-- (union would be better, but Coral8 doesn't support it) and get
-- the max of prices in any of the two windows.

create input stream StreamIn
schema (
   symbol string,
   price  float
);

-- Keep the events from the last 20 minutes.
create window Last20Mins
schema (
  symbol string,
  price  float
) keep 20 minutes
insert removed rows into Expired;

 -- Insert the events into the window.
insert into Last20Mins
select *
from StreamIn;

-- Keeps the previously expired event from Last20Mins (one per symbol)
create window Expired
schema (
  symbol string,
  price  float
) keep last per symbol;

-- For each symbol, store its max value over the last 20 minutes.
create window Result
schema (
  symbol string,
  result float
) keep last per symbol;

-- Join both windows and get the max of either of them
-- This would be prettier with UNION
--
-- Note: When events expire, this generates two events: one before
--       the expiration and another after.
insert into Result
select L.symbol,
       max(coalesce(max(L.price),0), coalesce(max(E.price),0))
from   Last20Mins as L full outer join Expired as E
         on L.symbol = E.symbol
group by L.symbol;
\end{lstlisting}

\section{The defective products problem}
\label{sec:defective-products-solution}

This is just a simplification of the original problem. In particular,
it doesn't handle humidity and assumes that rooms have a single
temperature sensor.

\begin{lstlisting}
-- Algorithm:
--
-- - Always maintain an explicit counter per product containing
--   the number of seconds the product spent at > 20 degrees.
--   This counter also contains the time the product was last
--   updated (last_update).
--
-- - When a product switches rooms, if the temperature in the
--   previous room was > 20, add now() + last_update to the
--   counter and replace the previous last_update with now();
--
-- - When the temperature in a room changes, if the previous
--   temperature was > 20, add now() + last_update to the
--   counter and replace the previous last_update with now();
--

create input stream entries
schema (
   product_id string,
   room_id string
);

create input stream temp_readings
schema (
   room_id string,
   temperature integer
);

-- Holds the last two entries per product
create window ProductRoom
schema (
   product_id string,
   room_id string
) keep 2 rows per product_id;

insert into ProductRoom
select *
from entries;

-- Holds the last two temp_readings per room
create window RoomTemperature
schema (
   room_id string,
   temperature integer
) keep 2 rows per room_id;

insert into RoomTemperature
select *
from temp_readings;

create local stream OnExit
schema (
   product_id string,
   old_room_id string
);

-- When a product leaves a room, generate an event containing the
-- product and the old room.
insert into OnExit
select entries.product_id, ProductRoom.room_id
from   entries left outer join ProductRoom
         on entries.product_id = ProductRoom.product_id
where  entries.room_id != ProductRoom.room_id;

create local stream OnTemperatureChange
schema (
   room_id string,
   old_temperature string
);

-- When the temperature changes, generate an event containing the
-- room and the old temperature
insert into OnTemperatureChange
select temp_readings.room_id, RoomTemperature.temperature
from   temp_readings left outer join RoomTemperature
         on temp_readings.room_id = RoomTemperature.room_id
where  temp_readings.temperature != RoomTemperature.temperature;

-- Per each product, maintain a counter with the time the product
-- spent above 20 degrees and the last time this counter was updated
create window ProductTemperatureCounter
schema (
   product_id    string,
   time_above_20 interval,
   last_update   timestamp
) keep last per product_id;

-- Initialize this counter
insert into ProductTemperatureCounter
select product_id, 0 seconds, now()
from entries
where room_id = "A";

-- Every time the product leaves the room, checks if the temperature
-- in the old room was > 20 and updates the counter of that product
insert into ProductTemperatureCounter
select E.product_id,
       if T.temperature > 20
         then time_above_20 + now() - C.last_update
         else time_above_20
       end,
       now()
from  OnExit as E, RoomTemperature as T,
      ProductTemperatureCounter as C
where E.old_room_id = T.room_id
        and E.product_id = C.product_id;

-- Every time the temperature changes, checks if the previous
-- temperature was > 20 and updates the counters of all the products
-- in that room
insert into ProductTemperatureCounter
select R.product_id,
       if T.old_temperature > 20
         then time_above_20 + now() - C.last_update
         else time_above_20
       end,
       now()
from  OnTemperatureChange as T, ProductRoom as R,
      ProductTemperatureCounter as C
where T.room_id = R.room_id
        and R.product_id = C.product_id;
\end{lstlisting}

\section{The Linear Road Benchmark in CQL}
\label{sec:lrb-cql}

\lstset{
  language=CQL,
  columns=fullflexible,
  basicstyle=\tt,
  keywordstyle=[1]\bf,
  keywordstyle=[2]\it,
}


\begin{lstlisting}
Query 1: SegSpeedStr(vehicleId, speed, segNo, dir, hwy)

  select vehicleId, speed, xPos/1760 as segNo, dir, hwy
  from PosSpeedStr

Query 2: ActiveVehicleSegRel(vehicleId, segNo, dir, hwy):

  select distinct L.vehicleId, L.segNo, L.dir, L.hwy
  from SegSpeedStr [range 30 seconds] as A,
        SegSpeedStr [partition by vehicleId rows 1] as L
  where A.vehicleId = L.vehicleId

Query 3: VehicleSegEntryStr(vehicleId, segNo, dir, hwy)

  select istream(*) from ActiveVehicleSegRel

Query 4: CongestedSegRel(segNo, dir, hwy)

  select segNo, dir, hwy
  from SegSpeedStr [range 5 minutes]
  group by segNo, dir, hwy
  having Avg(speed) < 40

Query 5: SegVolRel(segNo, dir, hwy, numVehicles)

  select segNo, dir, hwy, count(vehicleId) as numVehicles
  from ActiveVehicleSegRel
  group by segNo, dir, hwy

Query 6: TollStr(vehicleId, toll)

  select rstream(E.vehicleId,
                    basetoll * (V.numVehicles - 150)
                              * (V.numVehicles - 150) as toll)
  from VehicleSegEntryStr [now] as E,
       CongestedSegRel as C, SegVolRel as V
  where E.segNo = C.segNo and C.segNo = V.segNo and
         E.dir = C.dir and C.dir = V.dir and
         E.hwy = C.hwy and C.hwy = V.hwy
\end{lstlisting}


\section{MACD calculation in StreamSQL}
\label{sec:macd}

\lstset{
  language=StreamSQL,
  columns=fullflexible,
  basicstyle=\tt,
  keywordstyle=[1]\bf,
  keywordstyle=[2]\it,
}

\begin{lstlisting}
create input stream stocks (
    symbol string,
    price double,
    date timestamp
);
create output stream OutputAll;
create output stream OutputDELL;
create output stream OutputIBM;

create memory table SequenceTable
(
    sequence_name string,
    count int
) primary key(sequence_name) using hash;

create stream out__AddSequenceNumber_1;

insert into SequenceTable (sequence_name, count)
select "sequence" as sequence_name, 1 as count
from stocks
on duplicate key update
count = SequenceTable.count + 1
returning stocks.symbol as symbol,
          stocks.price as price, stocks.date as date,
          SequenceTable.count as count
into out__AddSequenceNumber_1;

create stream out__EMA_1 ;
select symbol as symbol,
       exp_moving_avg(price, 26,  0.0741) as EMaslow,
       exp_moving_avg(price, 12, 0.15385) as EMAFast,
       lastval(count) as count,
       lastval(date) as date,
       lastval(price) as price
from out__AddSequenceNumber_1[tuples valid always offset 0]
group by symbol
into out__EMA_1;

create stream out__ValidateEMA_1;
select * from out__EMA_1
where (notnull(EMaslow) && notnull(EMAFast)) into out__ValidateEMA_1;

create stream out__MACD_1;
select symbol as symbol,
       EMAFast as EMAFast,
       EMaslow as EMaslow,
       EMAFast - EMaslow as MACD,
       count as count,
       date as date,
       price as price
 from out__ValidateEMA_1
 into out__MACD_1;

create stream out__AggregateSignal_1;
select symbol as symbol, lastval(price) as price, lastval(MACD) as MACD,
       exp_moving_avg(MACD, 9, 0.2) as signal, lastval(date) as date
from out__MACD_1[tuples valid always offset 0]
group by symbol
into out__AggregateSignal_1;

select * from out__AggregateSignal_1
where notnull(signal) into OutputAll;

select * from OutputAll
where (symbol=="DELL") into OutputDELL;

select * from OutputAll
where (symbol=="IBM") into OutputIBM;


\end{lstlisting}