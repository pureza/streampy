% TODO
% DO NOT FORGET THE REFERENCE TO CORAL8 REFERENCE GUIDE
% Make a language reference

\documentclass[a4,11pt]{article}


\usepackage{alltt}
\usepackage{graphicx}
\usepackage{color}
\usepackage[small,bf]{caption}
\usepackage{listings}
\usepackage{cite}
\usepackage{url}

\begin{document}

\addtolength{\parskip}{\baselineskip}

\section*{Resumo}

Processamento de streams de eventos \'{e} uma classe de
aplica\c{c}\~{o}es que revela um enorme potencial na resolu\c{c}\~{a}o
de alguns problemas reais. Este tipo de aplica\c{c}\~{o}es
caracteriza-se por lidar com dados que est\~{a}o constantemente a
chegar para serem processados o mais depressa poss\'{i}vel, de modo a
produzir novos e melhores resultados. Existem in\'{u}meros
cen\'{a}rios que se adequam na perfei\c{c}\~{a}o a este modelo:
an\'{a}lise financeira, monitoriza\c{c}\~{a}o do estado de sa\'{u}de
de pacientes, detec\c{c}\~{a}o de intrus\~{o}es numa rede de
computadores, localiza\c{c}\~{a}o de pessoas ou produtos usando
tecnologia RFID, monitoriza\c{c}\~{a}o dos processos de neg\'{o}cio e
muitos mais. Actualmente j\'{a} existem sistemas gen\'{e}ricos que
podem ser adaptados a muitos dom\'{i}nios. Infelizmente, estes
sistemas disponibilizam ao utilizador linguagens de
programa\c{c}\~{a}o limitadas que colocam barreiras ao desenvolvimento
de aplica\c{c}\~{o}es um pouco mais complexas. Nesta tese tentamos
compreender quais s\~{a}o estas limita\c{c}\~{o}es atrav\'{e}s da
an\'{a}lise de alguns exemplos t\'{i}picos. Em seguida propomos uma
linguagem de programa\c{c}\~{a}o alternativa que \'{e}, na nossa
opini\~{a}o, capaz de suportar problemas mais complexos, sem que o
c\'{o}digo deixe de manter atributos de qualidade requeridos em
Engenharia de Software.


% Event Stream Processing (ESP) is a class of applications that shows
% potential to help solve many real-world problems. ESP applications are
% characterized by dealing with a possible infinite amount of data
% constantly flowing in to be processed as fast as possible to
% continuously produce new and updated results that may themselves be
% used to justify new decisions. It turns out that many applications fit
% naturally in this model: financial analysis, health-care monitoring,
% network intrusion detection, personnel and product tracking through
% RFID devices, business monitoring and many more. Unfortunately,
% existing ESP engines can be programmed using languages that are not

\end{document}

