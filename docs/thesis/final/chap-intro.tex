\chapter{Introduction}
\label{chap:introduction}

\section{Motivation}

Technology development and its widespread adoption over the last
decades has significantly increased the demand for information
processing systems. Not only do we now need to process larger amounts
of information coming from everywhere, we must do it faster as
well. For some companies, obtaining results a few milliseconds earlier
may be a significant advantage over the competition. For others,
however, reacting immediately is of critical importance, as it
happens, for example, in the case of security breaches or nuclear
power plant malfunctions.

One particular class of applications, now referred to as Event Stream
Processing (ESP), has been the subject of much attention over the last
few years due to the potential it presents to solve many real-world
problems. ESP applications are characterized by dealing with a
possible infinite amount of data constantly flowing in to be processed
as fast as possible to continuously produce new and updated results
that may themselves be used to justify new decisions. It turns out
that many applications fit naturally in this model: financial
analysis, health-care monitoring, network intrusion detection,
personnel and product tracking through RFID devices, business
monitoring and many more.

Due to the inability of existing technology to satisfy the increasing
demands of all these markets, computer scientists developed the first
Data Stream Management Systems (DSMS) \cite{stream} \cite{aurora}
\cite{telegraphcq}. Coming mostly from the database community, these
researchers intended to build a generic engine that abstracted away
all the low level details of managing streams of data in high
demanding scenarios, while retaining much of the querying capabilities
of regular Database Management Systems (DBMS), so that they could be
easily adapted to a multitude of domains. It's no surprise then, that
the first DSMSs inherited a SQL dialect with some new
extensions. Aurora \cite{aurora}, with its boxes and arrows graphical
queries, was the exception, but the operators it provided still took
inspiration from SQL. Later, when the first DSMS hit the market, some
emphasis was put on end-user interaction and some applications began
to include rule-based systems that allow the developer to specify how
he wants to react to events using Event Condition Action (ECA) rules,
a concept developed in the context of active databases
\cite{adbms-manifesto}. At the same time DSMS began to incorporate
features from Complex Event Processing (CEP) systems that allowed the
user to detect complex patterns and correlations among the input
streams of data. This required adding new constructs into the query
language. Recently, some companies unhappy with declarative, SQL-like
languages (see, for example, \cite{sql-impendance-mismatch:post} or
\cite{flexstreams-whitepaper}), began to support procedural languages,
more familiar to C and Java programmers. Nowadays, each product
includes its own flavor of a SQL-like language with its own unique
extensions, a rule-based language, a procedural language, or any
combination of these three. Besides the obvious issue created by the
lack of a standard, all this variety demonstrates that DSMS
applications have their own needs and a satisfactory end-user query
language for them is yet to be found.

To make matters worse, the semantics on many of these products
disagree in fundamental ways. In \cite{towards_stream_sql_standard},
researchers analyzed how two of the most prominent products from
Oracle and StreamBase react in the presence of simultaneous
events. Surprisingly, they found that, in some scenarios, not only do
their results diverge, but they also differ from the expected
answer. Furthermore, they concluded that this disparity may be blamed
on the semantics employed by each product. It also happens that many
products consistently implement semantics that, despite working
correctly for most problems, turn out to be inappropriate for others,
making some queries difficult, if not impossible, to write. We will
analyze a few examples where this happens later in chapter
\ref{chap:simple-questions-complex-answers}.

Nonetheless, more and more organizations are adopting DSMS and relying
on them to process data coming from everywhere, including core
business processes. As a consequence, ESP applications tend to grow
and become more complex. However, we believe that languages available
in existing products are not prepared for this kind of ``programming
in the large'', because they lack essential features that allow code
to mature without becoming unmaintainable.

\section{Goals}

In this thesis we will highlight the problems with existing languages
through concrete examples and, armed with this knowledge, we will then
propose a new query language for ESP systems and apply it to a few
realistic scenarios. Hopefully, our language will not suffer from the
same problems and will prove to be more expressive and user-friendly
than the alternatives.

Designing a programming language from scratch is a daunting task,
though, so it was necessary to set up a few limits. In particular, we
have not defined the semantics using formal methods, something which
is a requirement in today's programming language research
field. Instead, we opted to develop an experimental prototype -- a
decision that, we believe, was correct, considering that ESP languages
are very different from traditional ones. We have also opted to leave
out a few useful features from existing languages, not because they
wouldn't still be useful, but because they don't bring much to the
table. We favoured new constructs instead of the old ones, because
this helps to differentiate our language from the others. For this
reason, our prototype does not support important operators such as
\verb=join=, despite the fact that their integration would be
relatively straightforward. Finally, we have largely ignored
optimization concerns -- a dangerous issue when you consider how
different our language is.

\section{Contributions}

The contributions of this work are twofold. First, we have shown that
existing languages need to be improved, in order to cope with
increasing demand for solutions to harder and harder problems. Second,
we have presented an alternative language that, we believe, is in the
right direction, as it is able to handle some of these harder problems
through code that is understandable, reusable and, most of all,
maintainable.

\section{Outline}

The rest of this thesis is organized as follows:

\begin{itemize}
\item In chapter \ref{chap:soa}, we will discuss the state of the art
  pertaining to ESP languages;
\item Chapter \ref{chap:simple-questions-complex-answers} presents a
  few problems where currently available languages show
  limitations. We will also discuss how they could be fixed;
\item Our proposal --- EzQL ---, will then be introduced in chapter
  \ref{chap:ezql};
\item Chapter \ref{chap:implementation} will discuss the inner-workings
  of the prototype;
\item Evaluation of our language using common event processing
  problems takes place in chapter \ref{chap:eval};
\item Finally, conclusions and future work will be the topic of
  chapter \ref{chap:future-work}.
\end{itemize}
